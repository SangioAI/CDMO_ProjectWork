\documentclass{article}
\usepackage{amsmath}
\usepackage{float}

\title{\textbf{Project Title}}
\author{Marco Sangiorgi marco.sangiorgi24@studio.unibo.it \\
Andrea Fossà andrea.fossa@studio.unibo.it }

\begin{document}
\maketitle
Please do not be verbose. Describe only the relevant information, and omit the unnecessary/redundant details. Use a high-level, mathematical language in your model description, as was done in your course material. Do not copy and paste any piece of code in the report.
The report should be written in LATEXusing \\documentclass\{article\} without changing fonts, font size and margins. The page limits is 12 pages with three optimization approaches and 15 pages with all the approaches, excluding the references. You do not need to reach the page limit. You are recommended to use www.overleaf.com/ to produce a shared LATEXdocument.

\section{Introduction}
You do not need to introduce the problem, nor the optimization methods that you are using to tackle the problem. After a proper introduction to your report, describe and formalize what is common to all the models (such as input param- eters, the objective variable and its bounds, pre-processing steps, . . . ). Refer to this section where necessary instead of copying and pasting text in the following sections.
Also, briefly explain how you split the work between the group members, how long it took to complete the project and what the main difficulties have been.


\section{CP Model}
This part is mandatory for all groups.

\subsection{Decision variables}
Describe all the decision variables of your model, their initial domains (such as the lower and upper bounds), and their semantics. For example, “The variable Bi has the domain [0..100] with the meaning that Bi = j iff the baker i cooks j cakes, for i = 1,...,n.”

\subsection{Objective function}
If the objective variable and/or its bounds is specific to this model, describe here what is not covered in Section 1. Then explain the objective function. For example, “the objective function is to minimize area where area = Pni=1 Wi × Hi, because we need to minimize the total area. The bounds of the area variable were previously discussed in Section 1”.

\subsection{Constraints}

State the problem constraints, give their formulation and explain them. For example, “One constraint of the problem is that all locations should be dis- tinct. We impose this via the global constraint allDifferent([L1, . . . , Ln]) which enforces that the location variables Li for i = 1, . . . , n take different values.”
Start with the main problem constraints (i.e. the constraints that are strictly necessary) and then focus on the possible implied and symmetry breaking constraints that can help improve the model performance in the following subsections.


\paragraph{Implied constraints}
Optional, depending on the model formulation. In addition to the indication given in Section 2.3, discuss why the extra constraints are implied and how they could be useful.

\paragraph{Symmetry breaking constraints}
Optional, but highly recommended. In addition to the indication given in Section 2.3, describe which symmetries you observe in your model and discuss how the extra symmetry breaking constraints can reduce the observed symmetry.


\subsection{Validation}
The model must be implemented in MiniZinc and run using at least Gecode. Bonus points will be considered if other solvers are tried. The purpose of the validation is to assess the performance of the solvers using various models and search strategies. It is a good practice to evaluate also the contribution of the implied and symmetry breaking constraints.

\paragraph{Experimental design}
Before presenting your experimental results, give all the details of your exper- imental study. Your results should be reproducible following your description. Explain which solvers you used and which search strategies you employed, as well as your experimental set up (e.g., the hardware and the software used, any posed time limit etc). Irreproducible experiments will not be considered.

\begin{table}[H]
    \centering
    \begin{tabular}{ccccc}
         ID \vline& Chuffed + SB & Chuffed w/out SB & Gecode + SB & Gecode w/out SB \\\hline
         1  \vline& 100 & 120 & 80 & 80\\
         2  \vline& 50 & 60 & N/A & N/A \\
         3  \vline& UNSAT & UNSAT & N/A & N/A \\
    \end{tabular}
    \caption{Table 1: Results using Gecode and Chuffed with and without symmetry break- ing using the search strategy h.}
    \label{tab:my_label}
\end{table}


\paragraph{Experimental results}
Present your experimental results in a clear way. It is mandatory to show a table where:
\begin{itemize}
    \item rows are labeled with the identifier of the instances,
    \item columns are labeled with the different approaches you tried,
    \item cells contain the best objective value (not the runtime), as specified in the project description, found by the given approach on the given instance, using a certain search strategy. If the instance is solved to optimality, emphasise the objective value in bold. If the instance is proved to be unsatisfiable, indicate it as ’UNSAT’. If no answer is obtained within the time limit, indicate it with a ’N/A’ or ’-’.
\end{itemize}

For example, “The results obtained using the search strategy h are reported in Table 1”. The runtimes can be depicted using plots.


\section{SAT Model}
This part is mandatory for groups of 4 students. Groups up to 3 students can choose between SAT and SMT, and will be given bonus points if they do both.

\subsection{Decision variables}
Describe all the literals of your model and their semantics. For example, i,j = true iff the driver goes from city i to city j.

\subsection{Objective function}
See Section 2.2. Explain how you managed to do optimization in SAT.

\subsection{Constraints}
Describe all the clauses of your model and their semantics. In particular, de- scribe the encoding(s) that you used. Follow the indications given in Section 2.3 for main problem constraints, implied constraints, and symmetry breaking con- straints.

\paragraph{Implied constraints}

\paragraph{Symmetry breaking constraints}

\subsection{Validation}
See Section 2.4. The model must be implemented using at least Z3. Bonus points will be considered if a solver-independent language (e.g., Dimacs) is employed so as to play with different SAT solvers on the same model.

\paragraph{Experimental design}

\paragraph{Experimental results}


\section{SMT Model}
This part is mandatory for groups of 4 students. Groups up to 3 students can choose between SAT and SMT, and will be given bonus points if they do both.

\subsection{Decision variables}
See Section 2.1. Specify the sort of each variable and the theory used.

\subsection{Objective function}

\subsection{Constraints}
See Section 2.3.

\paragraph{Implied constraints}

\paragraph{Symmetry breaking constraints}

\subsection{Validation}
See Section 2.4.

\paragraph{Experimental design}

\paragraph{Experimental results}


\section{MIP Model}
This part is mandatory for all groups.

\subsection{Decision variables}
See Section 2.1.

\subsection{Objective function}
See Section 2.2. Note that the objective function must be linear.

\subsection{Constraints}
See Section 2.3. Note that each constraint must be linear.

\paragraph{Implied constraints}

\paragraph{Symmetry breaking constraints}

\subsection{Validation}
See Section 2.4. Bonus points will be considered if a solver-independent language (e.g., AMPL) is employed so as to play with different MIP solvers on the same model.

\paragraph{Experimental design}

\paragraph{Experimental results}


\section{Conclusions}
Write down briefly your concluding remarks.

\end{document}